% Trabalho de Política Habitacional
%
% Abaixo seguem orientações originais do modelo utilizado.
%
% Siga para o conteúdo do trabalho descendo até a linha 908.
%
% ==============================================================
%
% Modelo para monografia de final de curso, em conformidade
% com normas da ABNT implementadas pelo projeto abntex2.
%
% Este arquivo é fortemente baseado em exemplo distribuído no
% mesmo projeto. O projeto abntex2 pode ser acessado pela página
% http://abntex2.googlecode.com/
%
% Este arquivo pode ser rodado tanto com o pdflatex quanto com
% o lualatex.  Como contém referências bibliográficas a serem
% processadas pelo programa bibtex, este programa deve ser
% executado. Em resumo, a ordem de execução deve ser:
% rodar primeiro o pdflatex (ou o lualatex), depois o bibtex e,
% a seguir, o pdflatex (ou o lualatex ) novamente mais duas vezes,
% para assegurar que todas as referências bibliográficas e 
% citações estejam atualizadas.
%
% Para adaptar os textos para uso pessoal, usar os comandos
% imediatamente antes do \begin{document} (iniciando com o
% comando \titulo).  
%
% Este modelo está adaptado para monografias de final de curso
% em matemática da UFRJ, mas, com o uso das variáveis, pode ser
% usado para outros tipos de trabalho (mestrado, doutorado),
% outros cursos, universidades etc.  Caso a adaptação das
% variáveis não seja suficiente, pode-se alterar os comandos
% imprimircapa, imprimirfolhaderosto e imprimiraprovação, 
% fazendo as alterações necessárias.  Como os comandos definidos
% neste texto usam somente LaTeX, a sua adaptação deve ser 
% simples, bastando algum conhecimento de LaTeX.
%
% O restante do preâmbulo provavelmente  não necessitará ser
% alterado, a menos, eventualmente, das opções de chamada da
% classe abntex2, que estão definidas a seguir.
% 
\documentclass[ 
% -- opções da classe memoir que é a classe base da abntex2 --
% tamanho da fonte
12pt,
% capítulos começam em pág ímpar. Insere pág vazia, se preciso
openright,
% para imprimir uma página por folha ou visualização em video 
oneside,
% frente e verso. Margens das pag. ímpares diferem das pares.
%  twoside,
% tamanho do papel. 
a4paper,
% Caio - Ocultando bordas horríveis em hiperligações
hidelinks,
% -- opções da classe abntex2 --
% títulos de capítulos convertidos em letras maiúsculas
%  chapter=TITLE,
% títulos de seções convertidos em letras maiúsculas
%  section=TITLE,
% títulos de subseções convertidos em letras maiúsculas
%  subsection=TITLE,
% títulos de subsubseções convertidos em letras maiúsculas
%  subsubsection=TITLE,
% -- opções do pacote babel --
english,   % idioma adicional para hifenização
portuguese,   % o último idioma é o principal do documento
oldfontcommands,
]{abntex2}
%
% ==============================================================
%
% --------------------------------------------------------------
% Adicionando pacotes para recursos adicionais e defindo opções
% pertinentes
% --------------------------------------------------------------
%
% cabeçalho comum para uso com lualatex ou pdflatex
\usepackage{ifluatex}
% opções para uso com o lualatex
\ifluatex
\usepackage{fontspec}
\defaultfontfeatures{Ligatures=TeX}
% o fonte small caps é diferente no latin modern
\fontspec[SmallCapsFont={Latin Modern Roman Caps}]{Latin Modern Roman}
% pacotes da AMS 
\usepackage{amsmath,amsthm} 
% pacote para fonte específico para símbolos matemáticos
\usepackage{unicode-math}
\setmathfont{Latin Modern Math}
% latin modern tem simbolos de mathbb muito feios.
%  Trocar o fonte para estes simbolos.
\setmathfont[range=\mathbb]{Tex Gyre Pagella Math}
% opções para uso com o pdflatex
\else
\usepackage[utf8x]{inputenc}
\usepackage[T1]{fontenc}
\usepackage{lmodern}
\usepackage{etoolbox}
% pacotes da AMS 
\usepackage{amsmath,amssymb,amsthm} 
% Mapear caracteres especiais no PDF
\usepackage{cmap}
\fi

% pacotes usados tanto pelo lualatex quanto pelo pdflatex
\usepackage{lastpage}    % Usado pela Ficha catalográfica
\usepackage{indentfirst} % Indenta primeiro parágrafo 
\usepackage{color}       % Controle das cores
\usepackage{graphicx}    % Inclusão de gráficos
\usepackage{wrapfig}     % gráficos ao redor do texto
% pacote para ajustar os fontes em cada linha de forma a
% respeitar as margens
\usepackage{microtype}
% permite a gravação de texto em um arquivo indicado a partir
% deste arquivo.  Originalmente foi usado para criar o arquivo
% .bib com conteúdo de exemplo, evitando a edição de um arquivo
% .bib somente para a bibliografia
\usepackage{filecontents}

% Caio - diagramas
% http://www.texample.net/tikz/examples/smart-priority/
%\usepackage{smartdiagram}

% Caio - ladeando imagens
% https://tex.stackexchange.com/questions/57433/cannot-use-caption-under-minipage
\usepackage{caption}

% Caio - preciso de tabelas longas
% http://www.tex.ac.uk/FAQ-figurehere.html
\usepackage{longtable}

% Caio - quero alternar as cores das linhas da tabela
% https://tex.stackexchange.com/questions/107944/alternate-row-colors-in-longtable
\usepackage[table]{xcolor}
\definecolor{lightgray}{gray}{0.9}

% Caio - tentando melhorar o posicionamento das imagens
\usepackage{float}

% Caio - corrigindo espaçamento conforme http://tex.stackexchange.com/questions/5683/how-to-remove-top-and-bottom-whitespace-of-longtable
\setlength{\LTpre}{0pt}
\setlength{\LTpost}{0pt}

% Caio - preciso de plotagens
%\usepackage{pgfplots}
%\pgfplotsset{compat=1.8}

% Caio - quero usar letras nas listas do enumerate conforme https://tex.stackexchange.com/questions/2291/how-do-i-change-the-enumerate-list-format-to-use-letters-instead-of-the-defaul
\usepackage{enumitem}

% Caio - modo paisagem para tabelões
\usepackage{lscape}

% Caio - adicionando o pacote hyperref
\usepackage{hyperref}
% - e definindo metadados do PDF e comportamento dos links
\hypersetup{
	%pagebackref=true,
	pdftitle={Resenha do filme ``Um Fio de Esperança: Independência ou Guerra no Saara Ocidental''}, 
	pdfauthor={Vários},
	pdfsubject={Geografia Política},
	colorlinks=false,      		% false: boxed links; true: colored links
	linkcolor=blue,          	% color of internal links
	citecolor=blue,        		% color of links to bibliography
	filecolor=magenta,      	% color of file links
	urlcolor=blue,
	bookmarksdepth=4
}

% Caio - separação silábica
%\hyphenation{}

% Caio - citações mais poderosas
%\usepackage[autostyle]{csquotes}

%-----------------------------------------------------------
%-----------------------------------------------------------
% Caio - habilitar glossário
\usepackage{glossaries}
\makeglossaries

% \newglossaryentry{ex}{name={sample},description={an example}}
\newglossaryentry{onu}{
	name={ONU},
	description={Organização das Nações Unidas}
}

\newglossaryentry{minurso}{
	name={MINURSO},
	description={Missão das Nações Unidas para o referendo no Saara Ocidental}
}

\newglossaryentry{rasd}{
	name={RASD},
	description={República Árabe Saaraui Democrática}
}

\newglossaryentry{polisario}{
	name={Frente Polisário},
	description={Frente Popular de Liberación de Saguía el Hamra y Río de Oro}
}

%-----------------------------------------------------------
%-----------------------------------------------------------
% Comandos para definir ambientes tipo teorema em português 
\newtheorem{meuteorema}{Teorema}[chapter]
\newtheorem{meuaxioma}{Axioma}[chapter]
\newtheorem{meucorolario}{Corolário}[chapter]
\newtheorem{meulema}{Lema}[chapter]
\newtheorem{minhaproposicao}{Proposição}[chapter]
\newtheorem{minhadefinicao}{Definição}[chapter]
\newtheorem{meuexemplo}{Exemplo}[chapter]
\newtheorem{minhaobservacao}{Observação}[chapter]
%-----------------------------------------------------------
%-----------------------------------------------------------
% Pacotes de citações
\usepackage[brazilian,hyperpageref]{backref}
\usepackage[alf]{abntex2cite}   % Citações padrão ABNT
%\usepackage[num]{abntex2cite}  % Citações numéricas
% --- 
% Configurações do pacote backref
% Usado sem a opção hyperpageref de backref
\renewcommand{\backrefpagesname}{Citado na(s) página(s):~}
% Texto padrão antes do número das páginas
\renewcommand{\backref}{}
% Define os textos da citação
\renewcommand*{\backrefalt}[4]{
	\ifcase #1 %
	Nenhuma citação no texto.%
	\or
	Citado na página #2.%
	\else
	Citado #1 vezes nas páginas #2.%
	\fi}%
% --- 
% --- 
% Espaço em branco no início do parágrafo
\setlength{\parindent}{1.3cm}
% Controle do espaçamento entre um parágrafo e outro:
\setlength{\parskip}{0.2cm}  % tente também \onelineskip
% ---
% compila o indice, se este for incluído no texto
\makeindex
%
% --------------------------------------------------------- 
% ---------------------------------------------------------
% Redefinindo o comando do abntex2 para gerar uma capa  
\renewcommand{\imprimircapa}{%
	\begin{capa}
		\begin{flushleft} 
			{\centering \Large \textsc{\imprimirinstituicao  \\
					\imprimircurso \\} }
		\end{flushleft}
		
		\vfill
		\begin{center}
			{\large \imprimirautor} \\
			\vspace*{0.5cm}
			{\Large \textit{\imprimirtitulo}}
		\end{center}
		
		\vfill
		\begin{center}
			{\large{\imprimirlocal \\ \imprimirano  }}
		\end{center}
		\vspace*{1cm} 
	\end{capa}
	
}

% ---------------------------------------------------------
% ---------------------------------------------------------
%
%
% ---------------------------------------------------------
% ---------------------------------------------------------
% Redefinindo o comando para gerar uma folha de rosto 
\renewcommand{\imprimirfolhaderosto}{%
	\begin{center}
		{\large \imprimirautor}
	\end{center}
	\vfill \vfill \vfill \vfill
	\begin{center}
		{\Large \textit{\imprimirtitulo}}
	\end{center}
	
	\vfill \vfill \vfill 
	\begin{flushright} 
		\parbox{0.5\linewidth}{
			\imprimirtipotrabalho\, relacionado ao 
			\imprimircurso\, da \imprimirsigla\, 
			entregue como parte do
			processo de graduação para a obtenção do 
			grau de \imprimirgrau.}
	\end{flushright} 
	
	\vfill 
	\begin{flushright} 
		\parbox{0.5\linewidth}{ \imprimirorientadorRotulo 
			\imprimirorientador\\ \imprimirttorientador}
	\end{flushright} 
	
	\ifdefvoid{\imprimircoorientador}{}{
		\begin{flushright} 
			\parbox{0.5\linewidth}{ \imprimircoorientadorRotulo 
				\imprimircoorientador\\ \imprimirttcoorientador}
		\end{flushright}
	}
	
	\vfill \vfill \vfill \vfill \vfill \vfill \vfill
	\begin{center}
		{\large{\imprimirlocal \\ \imprimirano}}
	\end{center}
	\vspace*{1cm} \newpage
}
% Final do comando para gerar uma folha de rosto 
% ---------------------------------------------------------
% ---------------------------------------------------------
%
%
% ---------------------------------------------------------
% ---------------------------------------------------------
% Definindo o comando para gerar uma folha de defesa 
\newcommand{\imprimirfolhadeaprovacao}{%
	\begin{center}
		{\large \imprimirautor}
	\end{center}
	\vfill \vfill \vfill \vfill
	\begin{center}
		{\Large \textit{\imprimirtitulo}}
	\end{center}
	
	\vfill \vfill \vfill \vfill \vfill \vfill
	\begin{flushright} 
		\parbox{0.5\linewidth}{
			%			\imprimirtipotrabalho\,apresentada ao 
			%			\imprimircurso\, da \imprimirsigla\, como requisito
			%			para a obtenção parcial do grau de \imprimirgrau.}
		}
	\end{flushright} 
	\vfill \vfill \vfill \vfill
	Aprovada em \data.
	
	\vfill \vfill \vfill \vfill
	
	\begin{center}
		\textbf{BANCA EXAMINADORA}
		
		\vfill\vfill\vfill
		\rule{10cm}{.1pt}\\
		{\imprimirexaminadorum} \\ {\imprimirttexaminadorum}
		
		\ifdefvoid{\imprimirexaminadordois}{}{
			\vfill\vfill
			\rule{10cm}{.1pt}\\
			\imprimirexaminadordois \\ \imprimirttexaminadordois }
		
		\ifdefvoid{\imprimirexaminadortres}{}{
			\vfill\vfill
			\rule{10cm}{.1pt}\\
			\imprimirexaminadortres \\ \imprimirttexaminadortres }
		
		\ifdefvoid{\imprimirexaminadorquatro}{}{
			\vfill\vfill
			\rule{10cm}{.1pt}\\
			\imprimirexaminadorquatro \\ \imprimirttexaminadorquatro }
	\end{center}
	
	\vfill \vfill 
	\begin{center}
		{\large{\imprimirlocal \\ \imprimirano}}
	\end{center}
	\vspace*{1cm}
	\newpage
}
% Final do comando para gerar uma folha de defesa 
% ---------------------------------------------------------
% --------------------------------------------------------
%
%
%
%
%
% ---------------------------------------------------------
% --------------------------------------------------------
% definindo variáveis adicionais 
\providecommand{\imprimirsigla}{}
\newcommand{\sigla}[1]{\renewcommand{\imprimirsigla}{#1}}
%
\providecommand{\imprimircurso}{}
\newcommand{\curso}[1]{\renewcommand{\imprimircurso}{#1}}
%
\providecommand{\imprimirano}{}
\newcommand{\ano}[1]{\renewcommand{\imprimirano}{#1}}
%
\providecommand{\imprimirgrau}{}
\newcommand{\grau}[1]{\renewcommand{\imprimirgrau}{#1}}
%
\providecommand{\imprimirexaminadorum}{}
\newcommand{\examinadorum}[1]{
	\renewcommand{\imprimirexaminadorum}{#1}}
%
\providecommand{\imprimirexaminadordois}{}
\newcommand{\examinadordois}[1]{
	\renewcommand{\imprimirexaminadordois}{#1}}
%
\providecommand{\imprimirexaminadortres}{}
\newcommand{\examinadortres}[1]{
	\renewcommand{\imprimirexaminadortres}{#1}}
%
\providecommand{\imprimirexaminadorquatro}{}
\newcommand{\examinadorquatro}[1]{
	\renewcommand{\imprimirexaminadorquatro}{#1}}
%
\providecommand{\imprimirttorientador}{}
\newcommand{\ttorientador}[1]{
	\renewcommand{\imprimirttorientador}{#1}} 
%
\providecommand{\imprimirttcoorientador}{}
\newcommand{\ttcoorientador}[1]{
	\renewcommand{\imprimirttcoorientador}{#1}}
%
\providecommand{\imprimirttexaminadorum}{}
\newcommand{\ttexaminadorum}[1]{
	\renewcommand{\imprimirttexaminadorum}{#1}}
%
\providecommand{\imprimirttexaminadordois}{}
\newcommand{\ttexaminadordois}[1]{\renewcommand{
		\imprimirttexaminadordois}{#1}}
%
\providecommand{\imprimirttexaminadortres}{}
\newcommand{\ttexaminadortres}[1]{
	\renewcommand{\imprimirttexaminadortres}{#1}}
%
\providecommand{\imprimirttexaminadorquatro}{}
\newcommand{\ttexaminadorquatro}[1]{
	\renewcommand{\imprimirttexaminadorquatro}{#1}}
% fim da definição de variáveis adicionais
% ---------------------------------------------------------
% ---------------------------------------------------------
%
% ---
% ---
% ---
% ---
% ---
% ---
% ---
% ---
% ---
% Informações de dados para CAPA, FOLHA DE ROSTO e FOLHA DE DEFESA
%
%----------------- Título e Dados do Autor -----------------
\titulo{Resenha do filme ``Um Fio de Esperança: Independência ou Guerra no Saara Ocidental''}
\autor{Caio César C. Ortega \and
	Gabriel Alves \and
	Henrique Tutida \and
	Lorrayne Martinhão \and
	Nicholas Charles Bezerra
}
%

%----------Informações sobre a Instituição e curso -----------------
\instituicao{Universidade Federal do ABC \\
	Centro de Engenharia, Modelagem e Ciências Sociais Aplicadas}
%
\sigla{UFABC}
%
\curso{Bacharelado em Relações Internacionais}
%\curso{Curso de Licenciatura em Matemática}
%\curso{Mestrado em Ensino de Matemática}
%\curso{Doutorado em Matemática}
%
\local{São Bernardo do Campo, SP}
%
%
% -------- Informações sobre o tipo de documento
\tipotrabalho{Relatório}
%\tipotrabalho{Monografia de final de curso}
%\tipotrabalho{Dissertação de mestrado}
%\tipotrabalho{Tese de doutorado}
%
\grau{BACHAREL em Planejamento Territorial}
%\grau{LICENCIADO em Matemática}
%\grau{MESTRE em Matemática}
%\grau{DOUTOR em Ciências}
%
\ano{2018}
\data{13 de Junho de 2018} % data da aprovação
%
%------Nomes do Orientador, examinadores.  
\orientador{Rosana Denaldi}
%\coorientador{Antonio da Silva} % opcional
\examinadorum{Rosana Denaldi}
%\examinadordois{Ivo Fernandez Lopez}
%\examinadortres{Jeferson Leandro Garcia de Araújo}
%\examinadorquatro{Antonio da Silva}
%
%--------- Títulos do Orientador e examinadores ----
%\ttorientador{Bacharel em Física - UEFS}
%\ttcoorientador{Doutor em Matemática - UFRJ} 
%\ttexaminadorum{Doutor em Matemática - UFRJ}
%\ttexaminadordois{Doutor em Matemática - UFRJ}
%\ttexaminadortres{Doutor em Matemática - UFRJ}
%\ttexaminadorquatro{Doutor em Matemática - UFRJ}
%
% ---
% ---
\begin{document}
	
	% ---
	% Chamando o comando para imprimir a capa
	\imprimircapa
	% ---
	% ---
	% Chamando o comando para imprimir a folha de rosto
	%\imprimirfolhaderosto
	% ---
	% ---
	% Chamando o comando para imprimir a folha de aprovação
	%\imprimirfolhadeaprovacao
	% ---
	% ---
	% Dedicatória
	% ---
	%	\begin{dedicatoria}
	%  	 \vspace*{\fill}
	%  	 \centering
	%  	 \noindent
	%  	 \textit{ Este trabalho é dedicado a todos que, com entusiasmo,\\
	%  	 		sonham e lutam por XYZ no ABCDEFG\\
	%  			do XPTO.} \vspace*{\fill}
	%	\end{dedicatoria}
	%	
	%	
	%	\begin{agradecimentos}
	%	Orientação do modelo: insira aqui um parágrafo
	%	\end{agradecimentos}
	%	
	%	
	%
	%---------------------- EPÍGRAFE I (OPCIONAL)--------------
	%\begin{epigrafe}
	%    \vspace*{\fill}
	%    \begin{flushright}
	%        \textit{''Texto''\\
	%        Autor}
	%    \end{flushright}
	%\end{epigrafe}
	%
	%
	%
	%--------Digite aqui o seu resumo em %Português--------------
	%\begin{resumo}
	%   Descrição. 
	%
	%   \vspace{\onelineskip}
	%   \noindent
	%   \textbf{Palavras-chaves}: Palavras.
	%\end{resumo}
	
	
	%
	% --- resumo em inglês (abstract) ---
	%\begin{resumo}[Abstract]
	%   \begin{otherlanguage*}{english}
	%      Description.
	%
	%      \vspace{\onelineskip}
	%      \noindent
	%      \textbf{Keywords}: Words.
	%   \end{otherlanguage*}
	%\end{resumo}
	
	%
	%----Sumário, lista de figuras e lista de tabelas ------------
	\tableofcontents 
	%\newpage \listoffigures
	%\newpage \listoftables
	%---------------------
	%--------------Início do Conteúdo---------------------------
	% o comando textual é obrigatório e marca o ponto onde começa 
	% a imprimir o número da página
	\textual
	%
	%---------------------
	%
	
	
	%
	% O conteúdo começa pra valer a seguir
	%
	
	%
	%===============================================================================
	%
	
	% Endereço (URL) com as contribuições:
	% https://docs.google.com/document/u/2/d/143q36Y_-ceCrlNMS_yC0Ep0uHopwndZcm3CsPGDeEjI
	
	\chapter{Contextualização do filme} \label{cap:contextualizacao}
	% Responsável: Lorrayne (Lola)
	% Escopo: não é resumo
	
	O documentário ``Um fio de esperança'' é uma produção brasileira e independente com o intuito de difundir a causa do povo saarauí, que teve parte de seu território ocupado pelo Marrocos desde 1975 e espera por um referendo de autodeterminação há mais de 26 anos. Hoje, o Saara Ocidental é considerado a última colônia da África, embora a \gls{polisario} tenha proclamado a \glsdesc{rasd} (conhecida como \gls{rasd}) desde janeiro de 1976 que ganhou reconhecimento por um número limitado de outros países, mas não foi admitida na \glsdesc{onu}.
	
	Isso aconteceu porque logo após a Marcha Verde, movimento de ocupação ordenado por Hassan II, sultão do Reino do Marrocos na época, a Espanha abandonou sua antiga colônia sem organizar o referendo de autodeterminação, abrindo espaço para que o Saara fosse reclamado tanto pelo Marrocos quanto pela Mauritânia, que alegaram vínculos históricos como justificativa para a disputa do território. Mais tarde foi revelado também o Acordo de Madrid, que estipulava que a Espanha transferiria a administração do território para uma administração tripartite temporária, composta por Espanha, Marrocos e Mauritânia. Aí tem início uma guerra de 16 anos pela independência do Saara Ocidental.
	
	A \gls{polisario} surgiu como um movimento independentista político-revolucionário lutando pela autonomia do território desde 1973, ainda contra a ocupação espanhola, mas hoje representa ideologicamente a verdadeira organização política do povo saarauí, tanto em seu território ocupado quanto em campos de refugiados. O documentário foca bastante na paciente espera dos saarauís pelo prometido referendo de autodeterminação, sempre adiado ou proposto de forma a beneficiar o Marrocos.
	
	Já o interesse marroquino no território é profundamente econômico e se encontra nas riquezas naturais, que atualmente são exploradas sem ressalva pelo Marrocos. Esse interesse estende-se também à França, Estados Unidos e Espanha, que mantém relações econômicas próximas do Reino do Marrocos e se beneficiam da extração da riqueza do território do Saara. Nesse contexto, temos a principal razão pelo pouco esforço da \gls{onu} em efetivamente resolver a questão ou sequer garantir os direitos humanos do povo sarauí.
	
	Outra abordagem importante que o documentário faz é a respeito do posicionamento brasileiro, já que o Brasil é um dos únicos países da América Latina que não reconhecem oficialmente a \glsdesc{rasd}.
	
	\section{Ideias principais} \label{sec:ideias_principais}
	% Responsável: Henrique
	
	O documentário critica duramente a conduta da \glsdesc{onu} acerca do conflito envolvendo a \gls{polisario} com o Marrocos, e a situação precária em que se encontra o povo saarauí. A gênese do problema inicia-se em  1991, quando o Conselho de Segurança da \gls{onu} aprova a Resolução 690 por unanimidade, estabelecendo uma missão de paz no Saara Ocidental, a \glsdesc{minurso} (\gls{minurso}), que originalmente propunha garantir o cessar-fogo imediato entre ambos os lados; organizar e realizar um referendum, na qual a população escolheria entre ser incorporado pelo Marrocos ou se tornar independente; monitorar a posição das tropas marroquinas e polisárias; garantir que os direitos humanos fossem respeitados.
	
	Passados 25 anos, o cessar-fogo entre Marrocos e a \gls{polisario} foi o único objetivo alcançado pela \gls{minurso}. A \gls{onu} foi incapaz de realizar um referendum, pois o governo marroquino realizou diversas políticas de assentamento do Saara Ocidental, movendo cerca de 500 mil pessoas para o território que era originalmente ocupado pelo povo saarauí, impossibilitando a realização do referendum coeso de autodeterminação da população local. Houve a violação dos direitos humanos, com diversas denúncias de violência e desrespeito às populações civis saarauís nos territórios ocupados pelo Marrocos. A presença de 100 mil soldados marroquinas no Saara Ocidental e a construção de um muro de 2.700 km, marcou a incapacidade da \glsdesc{onu} em garantir a paz e a neutralidade do conflito.
	
	A França surge como um dos principais responsáveis para que o Saara Ocidental não se torne independente, seu poder de veto no Conselho de Segurança impede que haja qualquer intensificação da missão de paz, ou a imposição de sanções contra o Marrocos através da \gls{onu}. Além de apoiar militarmente e financeiramente o Marrocos no conflito.
	
	Dessa forma, o impasse de mais de duas décadas da \glsdesc{onu} sobre a independência do Saara Ocidental, somado às condições precárias em que a maior parte da população está submetida, têm levado a muitos saarauís em defender o retorno de um conflito armado para lutarem por sua independência. 
	
	\section{Ideias secundárias} \label{sec:ideias_outras}
	% Responsável: Nicholas
	
	O documentário explora o fato do conflito entre saarauís e marroquinos ser pouco conhecido e pouco mencionado no Brasil, mesmo entre estudantes de geopolítica. Entre deputados e senadores também não há tanta discussão sobre o tema. Há um grande destaque para a postura oficial brasileira sobre esses conflitos, que é de completa omissão. Mesmo com o apoio e ativismo de alguns parlamentares brasileiros a favor do povo saarauí, não há oficialmente o reconhecimento da independência do Saara Ocidental, algo já feito pela maioria dos países da América Latina.
	
	Além da omissão, que em alguns momentos é considerada cumplicidade, há denúncias do Brasil manter relações comerciais com o Marrocos, muitas vezes comprando produtos que são retirados ilegalmente da região do Saara Ocidental, como fosfato e sardinha. O apoio do Brasil é cobrado pela \gls{polisario}, que consideram nosso país como sendo muito importante. Nosso apoio traria mais peso e importância para a questão sem grandes impactos negativos, pois nossa balança comercial com o Marrocos é pequena.
	
	Outro ponto interessante do documentário é a forma que as mulheres saarauís se organizam e o que elas representam nessa sociedade. Elas são muito fortes e ocupam uma posição de protagonismo. É mencionado por um dos entrevistados que na cultura deles a mulher é importante e considerada sagrada, algo que infelizmente as torna alvos. Nos conflitos, os marroquinos tendem a agredir principalmente as mulheres, que é uma forma de atingir física e psicologicamente o povo saarauí.
	
	Muitos abusos aos direitos humanos são relatados, principalmente nas áreas ocupadas, o que deixa clara a omissão da \gls{minurso}. Para alguns membros da \gls{polisario} fica a impressão de que eles estão ali apenas para ganhar dinheiro e aproveitar, pois, mesmo com diversas denúncias e pedidos, os abusos seguem ocorrendo. Isso demonstra uma certa fragilidade na atuação das Organizações Internacionais para manutenção da segurança em alguns países.
	
	\chapter{Análise e discussão} \label{cap:analise}
	% Responsável: Caio
	% Escopo: análise e debate de pelo menos 2 autores discutidos em sala de aula com citações diretas e indiretas
	
	Como melhor discutido no \autoref{cap:contextualizacao} (\nameref{cap:contextualizacao}), o documentário apresenta a causa da \glsdesc{rasd}, sendo possível identificar pelo menos as seguintes abordagens: entrevistas e filmagens no território reclamado pela \gls{polisario}, incluindo as áreas ocupadas pelo Reino do Marrocos e entrevistas em Brasília, no Distrito Federal, com membros do parlamento e outros atores sensíveis à causa. Cabe salientar que não há uma versão do conflito por parte de quaisquer representantes do Marrocos, sendo que antes dos créditos, é fornecida a informação de que o embaixador não aceitou gravar após uma conversa de três horas, da mesma maneira, o Ministério das Relações Exteriores da República Federativa do Brasil só aceitava falar extra-oficialmente sobre o tema e o Ministério da Defesa, alegadamente participante de missões de paz da \gls{minurso}, não autorizou a entrevista de oficiais participantes de missões no Saara Ocidental.
	
	% Sugestão do Henrique:
	% Acho que daria pra relacionar com o expansionismo marroquino para o Saara Ocidental, com a questão de Ratzel sobre a importância de um grande território, e a mobilidade do território - migrações verdadeiras e politizadas.
	
	Analisando o documentário com vistas a relacionar o tema abordado com a bibliografia da disciplina, o primeiro ponto que pode ser apontado diz respeito à ligação entre o território e a população, ideia ligada à doutrina de Friedrich Ratzel (grifos nossos):
	
	\begin{citacao}
		``Este é também o caso de um artigo seu que aparece publicado na França em 1898, intitulado " O solo, a sociedade e o Estado", no qual o autor repõe a sua idéia geral de que o desenvolvimento estatal é processo que depende da \textbf{estreita ligação orgânica do	povo com o solo}, extraindo daí o seu \textbf{conceito de Estado como organismo territorial}.'' \cite[p. 40]{costa1992}
	\end{citacao}
	
	Considerando que segundo \citeonline[p. 81]{roseira2015}:
	
	\begin{citacao}
		``Essa geopolítica é guiada por uma visualização do mundo em blocos de espaço. Almeja sempre a divisão e ordenação do espaço global em blocos rivais de poder. O que dá coesão a esses blocos é a posição geográfica, as alianças políticas e comerciais, a política imperialista, os sistemas de circulação continental ou ultramarina etc.''
	\end{citacao}

	A postura do reino que rivaliza com a \gls{polisario} parece ser favorecida pela manutenção do \textit{status quo} pela \glsdesc{onu}, que no documentário é retratada como um ator incapaz de exercer pressão a países como França (que apoia o Marrocos) e Espanha (que contribuiu para iniciar o conflito em primeiro lugar e tem se omitido). Neste sentido a visão geopolítica ratzeliana é suficientemente conveniente para que possamos mencioná-la ao lançar luz sobre o embate. Há, portanto, um alinhamento do Marrocos com blocos de poder que contribuem para não permitir uma solução definitiva, acirrando o conflito entre as partes diretamente envolvidas e a agonia da população que se encontra vivendo nos assentamentos das chamadas ``zonas liberadas'', livres da presença militar do Marrocos e cercadas pelo muro já mencionado na \autoref{sec:ideias_principais} (\nameref{sec:ideias_principais}). Como ``as capitais e as fronteiras que também emergem, tal como foram concebidas, de códigos semânticos constituem articulações da linguagem da geografia do Estado'' \cite[p. 25]{raffestin1993}, é no mínimo sintomática a ocupação de El Aiune pelo Marrocos, visto que esta cidade é a capital da \glsdesc{rasd}, o que força a \gls{polisario} a organizar a capital provisoriamente em Tifariti.

	O documentário demonstra que apesar das características desfavoráveis da parcela do território \textit{de facto} sob domínio da \gls{polisario}, existe uma forte ideia de construção de um Estado com participação popular, protagonismo das mulheres e espírito combativo em defesa da identidade e autoafirmação independentista do povo saarauí. Ainda que militares e civis tenham externado desejo pela paz, há a crença de que retomar a luta armada pode ser a única forma de reivindicar a parcela do território ocupada pelo Marrocos, que tem explorado recursos naturais e obtido vantagens econômicas, inclusive praticando comércio com o Brasil.
	
	Outro aspecto que pode ser explorado, agora pela perspectiva marroquina, diz respeito à noção de expansionismo do autor, de corte malthusiano. Segundo \citeonline[p. 40--41]{costa1992}, na visão de Ratzel, o Estado ``deverá estar ciente de que o crescimento populacional, e conseqüentemente das necessidades de subsistência, pode criar sérios transtornos ao seu desenvolvimento''. Ora, ainda que o Marrocos não esteja temendo crises com a configuração original de seu território, excluindo a parcela do Saara Ocidental que suscita o objetivo, definitivamente os esforços envidados até agora indicam que aquele Estado está consciente de que promove um estrangulamento da população sobre tutela da \gls{polisario}, tanto que o documentário exibe não só reclamações da própria população, como também demonstra a necessidade de assistência externa e o boicote à inclusão econômica de indivíduos envolvidos com a \gls{polisario}, incluindo violações graves aos direitos humanos, como prática de cárcere em condições degradantes e tortura. Cabe também salientar que essa noção empregada por Ratzel se associa ao conceito de espaço vital, também empregado fortemente por Karl Haushofer\footnote{Espaço vital para Haushofer, conforme \citeonline[p. 139]{costa1992}:``Estratégia política para	os Estados, que leva em conta, necessariamente, a correspondência ideal entre a densidade populacional, os projetos de plena realização econômica e cultural das nações e a base territorial, indispensável ao pleno desenvolvimento de cada país.''}.
	
	Ainda com relação a Ratzel, pode-se também estabelecer um breve paralelo às ideias de coesão interna que este estabeleceu com relação à Alemanha de então e que, basicamente, se traduziam em soluções expansionistas (saída para o mar e pangermanismo europeu) \cite[p. 41]{costa1992}, e a postura do Marrocos, cuja ocupação do Saara Ocidental privilegia o acesso à maior parte da faixa litorânea, ampliando a costa do país e sua presença naquela parcela do Oceano Atlântico.
	
	Finalmente, o conceito de mobilidade das fronteiras\footnote{Segundo \citeonline[p. 37]{costa1992}, também chamada de ``mobilidade comandada por processos político-territoriais''} pode ser usado para refletirmos sobre a ocupação do Saara Ocidental pelo Marrocos, estimulando a presença de marroquinos nas cidades e parcelas do território pleiteadas pelos defensores da \glsdesc{rasd}. Este conceito de mobilidade também se atrela à guerra:
	
	\begin{citacao}
		``Nos estágios superiores, diz ele, dada a consolidação dos Estados e territórios rigidamente delimitados por fronteiras, essa forma complexa	de mobilidade só pode ocorrer mediante as guerras, [\dots]
		
		Para ele, a simples declaração de guerra faz desaparecer as fronteiras, estabelecendo-se um novo espaço de circulação referido a um `todo territorial'.''
	\end{citacao}
	
	\chapter{Crítica e considerações finais} \label{cap:final}
	% Responsável: Gabriel, mas Caio fez dois parágrafos
	
	\section{Crítica} \label{sec:critica}
	
	É sempre uma tarefa não muito grata  abordar temas complexos sobre os quais a maior parte dos interlocutores possui pouco entendimento. Se há, por um lado, o risco de a mensagem ser passada de maneira muito técnica, o que agradaria apenas a públicos específicos, de outro, existe a ameaça da superficialidade. Tarefa não grata deste documentário que, no fim das contas, se sai bem.
	
	Negligenciado pelo mundo, o conflito no Saara Ocidental já dura mais de 40 anos, sendo este país a última colônia africana. Desde a desocupação espanhola, o povo saarauí luta contra a ocupação ilegal por parte do Marrocos. A apatia por parte dos países do ocidente e das Organizações Internacionais pode levar a conflito armado.
	O documentário aqui analisado põe o espectador dentro do conflito, sempre de maneira muito crítica em relação à conduta brasileira a seu respeito. Seu tempo, curto, é muito bem aproveitado. A edição se concentra em dividir, de maneira bem coesa, todos os aspectos as serem tratados no longa. O documentário ``começa pelo final'', pelo desfecho da crítica dos autores, que se concentra na forma como o governo brasileiro se omite da situação e na forma como o assunto recebe pouca atenção no mundo. Em seguida, é estabelecido o cenário, o histórico do conflito é contado de maneira exitosa  com um roteiro bem escrito, detalhando a história saarauí desde as origens do atual impasse. São utilizadas entrevistas e imagens históricas, bem articuladas entre si, que correspondem bem ao roteiro.
	
	Impressiona também, por se tratar de uma produção independente, a forma pela qual o longa consegue prender o interlocutor à história. A trilha sonora ambientaliza muito bem aquilo que está sendo narrado e a fotografia, impecável, que, ao mesmo tempo que impressiona com as belezas do Saara, também nos deixa apreensivos com a forma crua de relatar a vida deste povo. 
	
	Após inserir-nos dentro da história e do cotidiano saarauí, ``Um Fio de Esperança'' constrói a imagem do que é a \glsdesc{rasd}. O documentário não nos apresenta um paraíso, onde existe um modelo de autogestão. Apresenta uma sociedade que sofre com a escassez, todavia consegue fazer um feito incrível, que é manter uma mínima organização. Há muita preocupação com o direitos das mulheres e com a educação, que recebe muito apoio de Cuba, o que contrasta com a posição omissa do Brasil. Aliás, contraste este feito de diversas formas durante o documentário.
	
	Por fim, contudo, há de se perceber um profundo ceticismo da população saarauí a respeito da \gls{onu}, do Brasil e o do ocidente de forma geral. Uma percepção de abandono, que para este povo significa que há apenas um fim: o confronto. Sentimento esse que o documentário consegue captar com êxito.
	
	\section{Considerações finais} \label{sec:final}
	
	% parte do Caio - início
	O documentário é bastante interessante, pois concilia o retrato da vida do povo saarauí e posicionamentos diversos, que vão de habitantes das zonas ocupadas e liberadas do Saara Ocidental a políticos e atores diplomáticos que transitam em Brasília. A compartimentação do documentário contribui para situar o telespectador nos momentos da narrativa e, em situações como a chegada em uma das zonas liberadas, em meio à penumbra da noite e em pleno deserto, tenta esgotar a linguagem cinematográfica para compartilhar o drama daqueles que ali vivem.
		
	Observa-se a dificuldade de maior visibilidade à causa, dentro e fora do Brasil, com a luta autodeterminada se arrastando por mais de duas décadas, enquanto o país que contribuiu majoritariamente para a existência do conflito em primeiro lugar se omite, enquanto a França, aliada do Marrocos, acaba por agravar a situação, dando um verniz silenciador em relação à postura então pacifista da \gls{polisario}.
	% parte do Caio - fim
		
	Não é fácil tratar de um assunto que, além de complexo, é negligenciado. Este documentário consegue superar essa dificuldade. De maneira direta e simples, porém acurada, consegue tratar assuntos complexos com muita seriedade, destacando a origem do conflito, os atuais envolvidos e as relações de poder no âmbito das relações internacionais que impedem uma solução justa e pacífica para o conflito no Saara Ocidental. Destarte, trata-se de uma verdadeira denúncia: o fato de, em pleno século XXI, ainda existir uma colônia africana e isso acontecer com o conhecimento e, diga-se, convivência das Organizações Internacionais e de países como o Brasil. 
	
	%===============================================================================
	%
	
	% ----------------------------------------------------------
	% ----------------------------------------------------------
	\postextual
	
	
	
	% informa o arquivo com a bibliografia. Deve ser o mesmo nome
	% (sem o sufixo) que será informado no ambiente filecontents
	% que está no final deste arquivo. Neste exemplo foi usado 
	% bibitemp.bib e bibtemp. Este comando insere a bibliografia
	% nesta posição (antes dos apêndices, anexos, índice remissivo)
	\bibliography{fontes}
	
	% ----------------------------------------------------------
	% Glossário
	% ----------------------------------------------------------

	% ---
	% Define nome e preâmbulo do glossário
	% ---
	\phantompart
	\renewcommand{\glossaryname}{Glossário}
	%\renewcommand{\glossarypreamble}{Esta é a descrição do glossário. Experimente visualizar outros estilos de glossários, como o \texttt{altlisthypergroup}, por exemplo.\\ \\}

	% ---
	% Traduções para o ambiente glossaries
	% ---
	\providetranslation{Glossary}{Glossário}
	\providetranslation{Acronyms}{Siglas}
	\providetranslation{Notation (glossaries)}{Notação}
	\providetranslation{Description (glossaries)}{Descrição}
	\providetranslation{Symbol (glossaries)}{Símbolo}
	\providetranslation{Page List (glossaries)}{Lista de Páginas}
	\providetranslation{Symbols (glossaries)}{Símbolos}
	\providetranslation{Numbers (glossaries)}{Números} 
	% ---
	
	% ---
	% Estilo de glossário
	% ---
	%\setglossarystyle{index}
	\setglossarystyle{altlisthypergroup}
	%\setglossarystyle{tree}
	
	
	% ---
	% Imprime o glossário
	% ---
	\cleardoublepage
	\phantomsection
	\addcontentsline{toc}{chapter}{\glossaryname}
	\printglossaries
	% ---
	
	% ----------------------------------------------------------
	% Apêndices e anexos
	% ----------------------------------------------------------
	% Consultar documentação do abnTeX2 caso deseje incluir apêndices e/ou anexos
	
\end{document}
